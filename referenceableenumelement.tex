%See README.md for required packages

%Defines command \thighlight if it is not already defined
\ifthenelse{\isundefined{\thighlight}}{\newcommand{\thighlight}[1]{#1}}{}

% tip by http://tex.stackexchange.com/a/73710/9075
\makeatletter
 \newcommand{\labelname}[1]{% \labelname{<stuff>}
  \def\@currentlabelname{#1}}%
\makeatother

% Setup environment
% Called ONCE for each enum
%
% #1: (optional) the seperator between EnumId and Number. Default: "-". If set to COMBINED, the output format changes to <PrintName> <EnumId><Number>
% #2: Enum Id (no spaces, only characters), also used later in defineReferenceableEnumElement
% #3: Print Name of enum type, if "ONLYSHORT" then only the short name is used and a long name never printed
\newcommand{\setupReferencableEnumElements}[3][-]{%
%Define global counter
\newcounter{#2}

%Store print name
\expandafter\newcommand\csname #2PrintName\endcsname{#3}

%Store separator
%tipp by http://tex.stackexchange.com/a/64019/9075
\expandafter\newcommand\csname #2Separator\endcsname{#1}

%Store COMBINED and setup cref
\newtoggle{#2Combined}
\ifthenelse{%
\equal{#1}{COMBINED}}{%
\toggletrue{#2Combined}
\crefformat{#2}{##2#2##1##3}
\Crefformat{#2}{##2#2##1##3}
}{%
\togglefalse{#2Combined}
\crefformat{#2}{##2#2#1##1##3}
\Crefformat{#2}{##2#2#1##1##3}
}

%Store ONLYSHORTCONFIG
\newtoggle{#2OnlyShort}
\ifthenelse{%
\equal{#3}{ONLYSHORT}}{%
\toggletrue{#2OnlyShort}
}{%
\togglefalse{#2OnlyShort}
}

%End of definition of setupReferencableEnumElements
}

%internal command, used only at defineReferenceableEnumElement
% #1: Enum Id (no spaces, only characters), also used to print short form of enum and to reference the setup at setupReferencableEnumElements
% #2: Full Name of the enum to be defined
\newcommand{\defineReferenceableEnumElementHelper}[2]{%
% \arabic{#1} returns the counter value
\iftoggle{#1OnlyShort}{%
#1\csname #1Separator\endcsname\arabic{#1}: #2%
}{%
\iftoggle{#1Combined}{%
\csname #1PrintName\endcsname\ #1\arabic{#1}: #2%
}{%
\csname #1PrintName\endcsname\ \arabic{#1} (#1\csname #1Separator\endcsname\arabic{#1}): #2%
}%
}%
}%

% Generic command to define counter elements
% Called for EACH element
%
% #1: (optional) the command to wrap the text under. Default thighlight
% #2: Enum Id (no spaces, only characters), also used to print short form of enum and to reference the setup at setupReferencableEnumElements
% #3: Full Name of the enum to be defined
% #4: Label Id of the enum when referencing it
\newcommand{\defineReferenceableEnumElement}[4][thighlight]{%
%increase counter
\refstepcounter{#2}%
%
%store the ``caption'' into a sepcial variable being used later for referencing
\labelname{#3}%
%
%label directly put after refstepcounter to enable sectioning commands to be used as enumeration
\label{enum:#2:#4}%
%
%Print out the thing
\csname #1\endcsname{\defineReferenceableEnumElementHelper{#2}{#3}}%
%
}


% Same as \defineReferenceableEnumElement but for inline referenceable elements!
% One uses either without ``Inline'' or with ``Inline'' for a group of types
\newcommand{\defineReferenceableEnumElementInline}[4]{%
 \refstepcounter{#2}%
 \thighlight{(#2\csname #2Separator\endcsname\arabic{#2}) #3}%
 \labelname{#3}%
 \label{enum:#2:#4}%
}


% Generic command to reference to an enum
% #1: Enum Id (no spaces, only characters!)
% #2: Label Id
\newcommand{\refEnumFull}[2]{%
 \refEnum{#1}{#2}
 (\enquote{\nameref{enum:#1:#2}})%
}


\newcommand{\refEnum}[2]{%
  \cref{enum:#1:#2}%
}
