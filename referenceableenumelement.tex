%Requires following packages
%  \usepackage{ifthen}
%  \usepackage{csquotes}
%  \usepackage{hyperref}

%Requires the macro \thighlight to be defined

% tip by http://tex.stackexchange.com/a/73710/9075
\makeatletter
 \newcommand{\labelname}[1]{% \labelname{<stuff>}
  \def\@currentlabelname{#1}}%
\makeatother

% Setup environment
% #1: (optional) the seperator between EnumId and Number. Default: "-".
% #2: Enum Id (no spaces, only characters), also used later in defineReferenceableEnumElement
% #3: Print Name of enum type
\newcommand{\setupReferencableEnumElements}[3][-]{%
%Define global counter
\newcounter{#2}

%Store print name
\expandafter\newcommand\csname #2PrintName\endcsname{#3}

%Store separator
%tipp by http://tex.stackexchange.com/a/64019/9075
\expandafter\newcommand\csname #2Separator\endcsname{#1}

%Setup cleveref: use the same format for both beginning of the sentence and in the middle of the sentence
\crefformat{#2}{##2#2#1##1##3}
\Crefformat{#2}{##2#2#1##1##3}
}

% Generic command to define counter elements
%
% Is not used directly, but by a wrapper element
%
% #1: Enum Id (no spaces, only characters), also used to print short form of enum and to reference the setup at setupReferencableEnumElements
% #2: Full Name of the enum to be defined
% #3: Label Id of the enum when referencing it%
\newcommand{\defineReferenceableEnumElement}[3]{%
 \refstepcounter{#1}%
%
% \arabic{#1} returns the counter value
 \thighlight{%
\ifthenelse{%
\equal{\csname #1PrintName\endcsname}{ONLYSHORT}}{%
#1\csname #1Separator\endcsname\arabic{#1}: #2%
}{%
\csname #1PrintName\endcsname\ \arabic{#1} (#1\csname #1Separator\endcsname\arabic{#1}): #2%
}%
}.
%
 %store the label into a sepcial variable being used later for referencing
 \labelname{#2}%
%
 \label{enum:#1:#3}%
}

% Same as \defineReferenceableEnumElement but for inline referenceable elements!
% One uses either without ``Inline'' or with ``Inline'' for a group of types
\newcommand{\defineReferenceableEnumElementInline}[3]{%
 \refstepcounter{#1}%
 \thighlight{(#1\csname #1Separator\endcsname\arabic{#1}) #2}%
 \labelname{#2}%
 \label{enum:#1:#3}%
}

% Generic command to reference to an enum
% #1: Enum Id (no spaces, only characters!)
% #2: Label Id
\newcommand{\refEnumFull}[2]{%
 \refEnum{#1}{#2}
 (\enquote{\nameref{enum:#1:#2}})%
}

\newcommand{\refEnum}[2]{%
  \cref{enum:#1:#2}%
}
