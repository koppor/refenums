% tip by http://tex.stackexchange.com/a/73710/9075
\makeatletter
 \newcommand{\labelname}[1]{% \labelname{<stuff>}
  \def\@currentlabelname{#1}}%
\makeatother

% Generic commant to define counter elements
% #1: Enum Id (no spaces, only characters!)
% #2: Print Name of enum type
% #3: Full Name
% #4: Label Id
% \arabic{#1} return the counter value
\newcommand{\defineReferenceableEnumElement}[4]{%
 \refstepcounter{#1}%
 \thighlight{#2 \arabic{#1} (#1-\arabic{#1}): #3}. %
 \labelname{#3}%
 \label{enum:#1:#4}%
}

% Same as \defineReferenceableEnumElement but for inline referenceable elements!
\newcommand{\defineReferenceableEnumElementInline}[4]{%
 \refstepcounter{#1}%
 \thighlight{(#1-\arabic{#1}) #3}%
 \labelname{#3}%
 \label{enum:#1:#4}%
}

% Generic command to reference to an enum
% #1: Enum Id (no spaces, only characters!)
% #2: Nummer des Elements oder definiertes Label
\newcommand{\refEnumFull}[2]{%
 \refEnum{#1}{#2}
 (\enquote{\nameref{enum:#1:#2}})%
}
\newcommand{\refEnum}[2]{%
 #1-\ref{enum:#1:#2}%
}

% NAMEN WERDEN GLOBAL VERWALTET
\newcommand{\enumCrawlerAnforderungen}{CA}
\newcounter{\enumCrawlerAnforderungen}
\newcommand{\defineCrawlerAnforderung}[2]{\defineReferenceableEnumElement{\enumCrawlerAnforderungen}{Anforderung}{#1}{#2}}

% VERWENDUNG
% Definieren des Referenceable Enum environments
 %\newcounter{<EnumId>}
 %\newcommand{\define<EnumId>}[1]{\defineReferenceableEnumElement{<EnumId>}{Anforderung}{#1}}
% Definieren von Elementen
 %\defineCrawlerAnforderung{ETG Quality} % Name
 %\label{enum:<EnumId>:<label>} % muss "enum" sein! 
% Referenzieren
 % \refEnumFull{<EnumId>}{<label>} = <EnumId>-<Counter> ("<fullName>") z.B. CA-1 ("Qualit�t")
 % \refEnum{<EnumId>}{<label>} = <EnumId>-<Counter> z.b. CA-1
 % Ganz normale Referenzen auf das definierte Label
 % \nameref{enum:<EnumId>:<label>} = <fullName>
 % \ref{enum:<EnumId>:<label>} = <Counter>, e.g. 1
