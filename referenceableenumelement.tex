%Requires \usepackage{ifthen}

% tip by http://tex.stackexchange.com/a/73710/9075
\makeatletter
 \newcommand{\labelname}[1]{% \labelname{<stuff>}
  \def\@currentlabelname{#1}}%
\makeatother

% Generic command to define counter elements
%
% Is not used directly, but by a wrapper element
%
% #1: Enum Id (no spaces, only characters), also used to print short form of enum
% #2: Print Name of enum type
% #3: Full Name of the enum to be defined
% #4: Label Id of the enum when referencing it
%
\newcommand{\defineReferenceableEnumElement}[4]{%
% \arabic{#1} returns the counter value
 \refstepcounter{#1}%
 \thighlight{#2 \arabic{#1} (#1-\arabic{#1}): #3}. %
 \labelname{#3}%
 \label{enum:#1:#4}%
}

% Same as \defineReferenceableEnumElement but for inline referenceable elements!
% One uses either without ``Inline'' or with ``Inline'' for a group of types
\newcommand{\defineReferenceableEnumElementInline}[4]{%
 \refstepcounter{#1}%
 \thighlight{(#1-\arabic{#1}) #3}%
 \labelname{#3}%
 \label{enum:#1:#4}%
}

% Generic command to reference to an enum
% #1: Enum Id (no spaces, only characters!)
% #2: Label Id
\newcommand{\refEnumFull}[2]{%
 \refEnum{#1}{#2}
 (\enquote{\nameref{enum:#1:#2}})%
}

\newcommand{\refEnum}[2]{%
 #1-\ref{enum:#1:#2}%
}
