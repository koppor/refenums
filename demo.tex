\documentclass{article}

%Always add a dot behind the definition
\newcommand{\refenumenclosing}[1]{#1.}
%
\usepackage{refenums}


\begin{document}

\tableofcontents
\clearpage

%Setup ``requirements''
\setupReferencableEnumElements{R}{Requirement}

%Setup ``capabilities'', without any separator
\setupReferencableEnumElements[]{C}{Capability}

%Setup ``Notes'' showing the usage of defineReferenceableEnumElementInline
\setupReferencableEnumElements{N}{Note}

%Setup ``Steps'' not having a print form
\setupReferencableEnumElements[ ]{Step}{ONLYSHORT}

%Setup ``Milestones'' where the print name and the short name is combined
\setupReferencableEnumElements[COMBINED]{M}{Milestone}


\section{Requirements}
\label{secreqs}
\defRefEnum{R}{Scalability}{sca}
We see scalability as important requirement.

\defRefEnum{R}{Portability}{port}
We also see portability as important requirement.


\section{Capabilities}
\defRefEnum{C}{Maintainability}{maint}

\section{Notes}
\defRefEnumInline{N}{Blue}{blue} It should be a blue note, shouldn't it?

\section{Steps}
\label{sec:steps}

\defRefEnum[subsection]{Step}{Requirements Analysis}{rqa}
\label{sec:rqa}
The heading is defined using the macro.

\section{Milestones}
\defRefEnum[subsection]{M}{Basic Model}{bm}

%to show that the links are working: start a new page
\clearpage


\section{Discussion}
In \cref{secreqs}, we discussed \refEnumFull{R}{sca} and \refEnumFull{R}{port}.

We showed the capability \refEnumFull{C}{maint}.

We also had a note \refEnumFull{N}{blue}.

In \cref{sec:steps}, we started with \refEnumFull{Step}{rqa}.
\LaTeX{} put that step into \cref{sec:rqa}.

First, the milestone \refEnumFull{M}{bm} has to be reached.

\end{document}